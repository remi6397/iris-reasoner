\documentclass{article}

\usepackage{hyperref}

\title{The IRIS datalog parser}
\date{\today}
\author{Richard P\"ottler}

\begin{document}

\maketitle
\tableofcontents

\section{What is datalog?}

Datalog is a database query language that syntactically is a subset of Prolog.
Its origins date back to around 1978 when Herve Gallaire and Jack Minker
organized a workshop on logicand
databases.\footnote{http://en.wikipedia.org/wiki/Datalog}

\section{The grammar}

IRIS evaluates programs which contains rules and facts (the knowlege base) and
queries to be evaluated against this knowlege base.

All rules, facts and queries must be terminated by a `\verb|.|'.
\begin{description}
\item[rules]
consists of a head and a body. Both, the head and the body, are lists of
literals where the literals are separated by `\verb|,|' and the head and the body
are separated by `\verb|:-|'. The `\verb|,|' means `and'.\\
E.g.~`\verb|ancestor(?X, ?Y) :- ancestor(?X, ?Z), ancestor(?Z, ?Y).|'
\item[facts]
are single literals. E.g.~`\verb|ancestor('john', 'odin').|'
\item[queries]
are literals prefixed with a `\verb|?-|'. At the moment only queries with
\underline{one} literal are allowed. E.g.~`\verb|?- ancestor('john', ?X).|'
\item[literals]
 have the format `\verb|<predicate-symbol>(<terms>)|'. The terms must be
separated by `\verb|,|'. E.g.~`\verb|ancestor('john', 'garfield')|'.
\item[terms]
are either constants or variables.
\item[variables]
are simple strings prefixed with a `\verb|?|'. E.g.~'\verb|?VAR|'.
\end{description}

\subsection{Writing constants}

The datatypes IRIS supports are descriped in the table \ref{datatypes} on page
\pageref{datatypes}.

\begin{table}[hbp]
\begin{tabular}{|l|l|}
\hline
\hline
datatype & syntax \\
\hline
\hline
string & \verb|'<string>'| \\
& \verb|_string('<string>')| \\
\hline
decimal & \verb"'-'?<integer>.<fraction>" \\
& \verb"_decimal('-'?<integer>.<fraction>)" \\
\hline
integer & \verb|'-'?<integer>| \\
& \verb|_integer('-'?<integer>)| \\
\hline
float &  \verb|_float(<integer>.<fraction>)| \\
\hline
double &  \verb|_double(<integer>.<fraction>)| \\
\hline
iri &  \verb|_'<iri>'| \\
&  \verb|_iri('<iri>')| \\
\hline
sqname &  \verb|<string>#<string>| \\
&  \verb|_sqname('<string>#<string>')| \\
\hline
boolean & \verb|_boolean(<string>)| \\
\hline
duration & \verb|_duration(<year>, <month>, <day>, <hour>, <minute>, | \\
& \hspace{1cm} \verb|<second>)| \\
& \verb|_duration(<year>, <month>, <day>, <hour>, <minute>, | \\
& \hspace{1cm} \verb|<second>, <millisec>)| \\
\hline
datetime & \verb|_datetime(<year>, <month>, <day>, <hour>, <minute>, | \\
& \hspace{1cm} \verb|<second>)| \\
& \verb|_datetime(<year>, <month>, <day>, <hour>, <minute>, | \\
& \hspace{1cm} \verb|<second>, <tzHour>, <tzMinute>)| \\
& \verb|_datetime(<year>, <month>, <day>, <hour>, <minute>, | \\
& \hspace{1cm} \verb|<second>, <millisec>, <tzHour>, <tzMinute>)| \\
\hline
date & \verb|_date(<year>, <month>, <day>)| \\
& \verb|_date(<year>, <month>, <day>, <tzHour>, <tzMinute>)| \\
\hline
time & \verb|_time(<hour>, <minute>, <second>)| \\
& \verb|_time(<hour>, <minute>, <second>, | \\
& \hspace{1cm} \verb|<tzHour>, <tzMinute>)| \\
& \verb|_time(<hour>, <minute>, <second>, <millisec>, | \\
& \hspace{1cm} \verb|<tzHour>, <tzMinute>)| \\
\hline
gyear & \verb|_gyear(<year>)| \\
\hline
gyearmonth & \verb|_gyearmonth(<year>, <month>)| \\
\hline
gmonth & \verb|_gmonth(<month>)| \\
\hline
gmonthday & \verb|_gmonthday(<month>, <day>)| \\
\hline
gday & \verb|_gday(<day>)| \\
\hline
hexbinary & \verb|_hexbinary(<hexbin>)| \\
\hline
base64binary & \verb|_base64binary(<base64binary>)| \\
\hline
\end{tabular}
\caption{All supported datatypes}
\label{datatypes}
\end{table}

\subsection{Using built-ins}

A built-in can be written instead of a literal and IRIS will do its best to
evaluate it. All supported builtins are described in table \ref{builtins} at
page \pageref{builtins}.

\begin{table}[hbp]
\begin{tabular}{|l|l|l|}
\hline
\hline
name & syntax & supported operations \\
\hline
\hline
add & \verb|?X + ?Y = ?Z| & $numeric + numeric = numeric$ \\
& \verb|ADD(?X, ?Y, ?Z)| & $date + duration = date$ \\
& & $duration + date = date$ \\
& & $time + duration = time$ \\
& & $duration + time = time$ \\
& & $datetime + duration = datetime$ \\
& & $duration + datetime = datetime$ \\
& & $duration + duration = duration$ \\
\hline
subtract & \verb|?X - ?Y = ?Z| & $numeric - numeric = numeric$ \\
& \verb|SUBTRACT(?X, ?Y, ?Z)| & $date - duration = date$ \\
& & $date - date = duration$ \\
& & $time - duration = time$ \\
& & $time - time = duration$ \\
& & $datetime - duration = datetime$ \\
& & $datetime - datetime = duration$ \\
& & $duration - duration = duration$ \\
\hline
multiply & \verb|?X * ?Y = ?Z| & $numeric - numeric = numeric$ \\
& \verb|MULTIPLY(?X, ?Y, ?Z)| & \\
\hline
divide & \verb|?X / ?Y = ?Z| & $numeric / numeric = numeric$ \\
& \verb|DIVIDE(?X, ?Y, ?Z)| & \\
\hline
equal & \verb|?X = ?Y| & $any type = any type$ \\
& \verb|EQUAL(?X, ?Y)| & \\
\hline
unequal & \verb|?X != ?Y| & $any type \ne any type$ \\
& \verb|UNEQUAL(?X, ?Y)| & \\
\hline
less & \verb|?X < ?Y| & $any type < same type$ \\
& \verb|LESS(?X, ?Y)| & \\
\hline
less-equal & \verb|?X <= ?Y| & $any type \le same type$ \\
& \verb|LESS_EQUAL(?X, ?Y)| & \\
\hline
greater & \verb|?X > ?Y| & $any type > same type$ \\
& \verb|GREATER(?X, ?Y)| & \\
\hline
greater-equal & \verb|?X >= ?Y| & $any type \ge same type$ \\
& \verb|GREATER_EQUAL(?X, ?Y)| & \\
\hline
\end{tabular}
\caption{All supported builtins}
\label{builtins}
\end{table}

There is also a possibility to register selfwritten built-ins in IRIS and then
parse them. Therefore, if you have written and registered a built-in with the
predicatesymbol `\verb|ATOI|' you write `\verb|ATOI(?MY_STRING, ?MY_INT)|'.

\end{document}
