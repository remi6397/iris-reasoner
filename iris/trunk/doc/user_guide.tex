\documentclass{article}

\usepackage{hyperref}

\title{IRIS - Integrated Rule Inference System - API and User Guide}
\date{\today}
\author{Richard P\"ottler}

\begin{document}

\maketitle

\pagebreak

\tableofcontents

\pagebreak

\section{Introduction}

\subsection{Purpose}

This document is intended to give a short introduction to using the
Integrated Rule Inference System (IRIS) and its application programming interface (API).

\subsection{Audience}

This guide is for software developers who will be integrating IRIS in to their application
as well as logicians/researchers who wish to understand the capabilities of the IRIS reasoner.

\subsection{Scope}

The creation of logic programs and their evaluation is described.
The logic program can be created in one of two ways: Using the API
to construct a program fragment by fragment, or using the parser to
process a logic program written in human-readable format (datalog).

However, this document does not attempt to explain the theory of
logic programming and only provides a brief description of the
evaluation strategies employed.

\section{Description}

\subsection{What it does}

IRIS is a datalog reasoner that can evaluate safe-datalog with
stratified negation as failure.

It is delivered in two java `jar' files. One of which contains
the reasoner and the other contains the parser.

\subsection{What is input}

IRIS evaluates queries over a knowledge base.
The knowledge base consists of facts (instances of predicates) and rules.
The combination of facts, rules and queries is known as a logic program
and forms the input to a reasoning (query-answering) task.

The creation of the logic program is achieved in one of two ways:
Create the java objects representing the components of the program using the API
(described in section \ref{api} on page \pageref{api})
or parse an entire datalog program written in human-readable form using the parser.
The grammar supported by IRIS is described in the datalog grammar guide in section
\ref{parser} on page \pageref{parser}).

\subsection{What is output}

For each query in the datalog program, IRIS will return the variable bindings,
i.e. the set of all tuples that can be found or inferred from the knowledge base that satisfy the query.

\section{Evaluation Process}

\subsection{Supported Strategies}

A number of evaluation strategies are supported and it is intended that
more strategies will be created over time.

\subsubsection{Naive}

`Naive' evaluation is a bottom up approach where all the known facts are
applied in turn to each of the rules and thus new facts are inferred.
The evaluation process continues iteratively until no new facts are generated.
In this way the fixed point for the knowledge base is computed
before searching for the variable substitutions for the query.

\subsubsection{Semi-Naive}

The `Semi-Naive' algorithm is similar to the `Naive' strategy. However, in
every iteration an attempt is made to make better use of the tuples generated
in the previous step.

Consider a rule with two literals:
\begin{verbatim}
p(?X) :- q(?X), r(?X)
\end{verbatim}

In the first round the whole relation for `q' is joined with the whole relation
for `r' to produce new tuples for the relation `p'.

During this iteration, other rules might also generate new tuples for `q' and
`r', such that the rule must be evaluated again in the next iteration. However,
in an attempt to avoid generating tuples for `p' that are already known, the
join is made twice using only the incremental tuples from the previous
iteration.

Iteration 1: New tuples for p, q, r are generated: $\delta{}p_{1}$, $\delta{}q_{1}$, $\delta{}r_{1}$

Iteration 2: New tuples are generated for the rule above by joining $\delta{}q_{1}$ and r
(the whole relation for `r') and by joining q and $\delta{}r_{1}$

In other words, in each iteration a rule with N literals labeled $l_{1-N}$ is evaluated
N times substituting the incremental relations from the previous step for each each literal in turn.

\subsubsection{Semi-naive with Magic Sets}
 
The `Magic-Sets' strategy first transforms the rules according to the variable
bindings of the query so that the computation of the fixed point is restricted as
much as possible to those tuples that `might' produce something that satisfies the query.
(see `On the Power of Magic'\footnote{http://portal.acm.org/citation.cfm?id=28659.28689}
by Beeri and Ramakrishan).
After the transformation the `Semi-Naive' evaluation is performed.

\section{What can go wrong}

\subsection{Exceptions}

A number of problems can occur that can halt the evaluation of a logic program.
These problems are indicated by throwing an exception of one of the following types:

\begin{description}
\item[EvaluationException] is the superclass of all exceptions that halt
the evaluation process.
\item[ProgramNotStratifiedException] indicates, that the program is not
stratified (see \ref{stratneg} on page \pageref{stratneg}).
\item[RuleUnsafeException] indicates, that an unsafe rule was detected
(see \ref{safe} on page \pageref{safe}).
\end{description}

\subsection{Stratified negation}
\label{stratneg}
The 'meaning' of negation in logic programs has been discussed at length in literature.
Here we adopt the relational model and describe the following construct:
\begin{verbatim}
p(X) :- q(X), not r(X)
\end{verbatim}

as meaning, that the relation associated with predicate `p' contains all those values
from predicate `q' that are not in predicate `r'.
In other words, the set difference of `q' and `r'.

Traditional forward chaining methods for evaluating logic programs involve simply using
the values of tuples from predicates and applying them to the program's rules to generate
more tuples.

Without negation such techniques are guaranteed to be monotone.
However, in the presence of negation, rules that generate tuples for a predicate
that is used in negated sub-goals of other rules, must be 'fully' evaluated before
evaluation of the dependant rules begins.

Consider what would happen if we have the following:
\begin{verbatim}
p(X) :- q(X), not r(X)	...(1)
r(X) :- t(X)	        ...(2)
q(a)
q(b)
t(a)
\end{verbatim}

If the known facts are applied to rule (1) first, the following new facts are generated:
\begin{verbatim}
p(a)
p(b)
\end{verbatim}

Then applying the known facts to rule (2) produces the following:
\begin{verbatim}
r(a)
\end{verbatim}

However, the existence of fact r(a) should have precluded the inference of fact p(a) in rule (1).

In order to ensure that rule evaluation is monotone, rules must be evaluated in a specific order.

For any general rule:


$p :- L_{1}...L_{m}, N_{1}...N_{p}$


where $L_{1}...L_{m}$ are positive literals and $N_{1}...N_{p}$ are negative literals,
existing stratification algorithms require that the rule `p' be allocated to a strata
that is at least as high as each of its positive literals and at least one higher than
each of its negative literals.

Such a scheme would therefore require that rule (2) above is evaluated before rule(1).

However, this approach precludes the evaluation of any logic program containing a rule that
has a negative dependency upon itself.

In order for IRIS to evaluate a logic program, it must be stratified.

\subsection{Locally stratified negation}
\label{localstratneg}
There are genuine reasoning activities that can lead to the creation of logic programs
containing rules that do contain a negative dependency to themselves,
but can still be evaluated in a meaningful way,
because of the presence of constants in the rules that separate the domains of tuples
used as input to the rule and tuples produced by the rule.

Consider:
\begin{verbatim}
p(a,X) :- q(X), not p(b,X)
\end{verbatim}

This rule can produce tuples (a,?) for the relation associated with predicate `p'
from tuples (b,?) also associated with the relation for `p'.
However, no special treatment is required, because nothing produced by the rule can be
used as input to the rule, because of the presence of constants `a' and `b'.

A more complicated scenario is as follows:
\begin{verbatim}
p(a,X) :- r(X), not q(b,X)
q(X,Y) :- p(X,Y)
\end{verbatim}

Here the second rule can produce tuples for input to the first rule and vice versa.
However, if we consider the second rule to be two rules:
\begin{verbatim}
q(b,Y) :- p(b,Y)
q(X,Y) :- p(X,Y), X != b
\end{verbatim}
Then we see that the complete set of rules is still stratified.


\subsection{Unsafe rules}
\label{safe}

The algorithm for detecting unsafe rules is taken from `Principles of Database
and Knowledgebase Systems', Ullman, page 105.
A rule is safe if all the variables occurring in the head and body are limited.
Limited means, that the variable appears at least once in a positive ordinary predicate,
is equated with a constant in a positive equality predicate, is equated with
another variable known to be limited or occurs in an arithmetic predicate where
the other two variables are limited.
A more precise explanation of safeness can be found on the
homepage\footnote{http://www.iris-reasoner.org/saferules}



\section{Datatypes and Built-in Predicates}

\subsection{Supported datatypes}
\label{data}

IRIS supports the datatypes defined in the WSML specification\footnote{http://www.wsmo.org/TR/d16/d16.1/v0.21/\#sec:wsml-builtin-datatypes},
which are a subset of the XML schema datatypes.

These datatypes are also discussed in the appendix \ref{grammar_datatypes} on page \pageref{grammar_datatypes}.

\subsection{Built-in Predicates}
\label{builtins}

The built-in predicates defined in the WSML
specification\footnote{http://www.wsmo.org/TR/d16/d16.1/v0.21/\#sec:wsml-built-ins}
are supported plus some IS\_$<$type$>$() predicates used to confirm the type of a variable.

The complete list of built-in predicates is given in the appendix \ref{grammar_builtins} on page \pageref{grammar_builtins}).

Additionally, user-defined built-in predicates
can be created (see \ref{custom_builtins} on page \pageref{custom_builtins}).


\subsection{Behaviour of built-ins with incompatible datatypes}
\label{naf}

Built-in predicates will evaluate to false if the operands are incompatible with the predicate,
e.g. multiplying two dates.

Built-in predicates will evaluate to false if the operands are incompatible with each other,
e.g. adding an integer to a string.


\subsection{Negated built-ins}

Negation in IRIS means `negation as failure', so the meaning of the expression `p(X) and not q(X)'
is the relation containing every value of X for which p() is true, removing every value of X for which q() is true.
In this context, care must be taken when using negation with built-in predicates. Consider the following program:
\begin{verbatim}
p(1,2).
p(2,3).
p(4,3).
p(`a',4).

q(x,y) :- p(x,y), x >= y.

?-q(x,y).
\end{verbatim}
This produces the result set:
\begin{verbatim}p(4,3)\end{verbatim}
However this program:
\begin{verbatim}
p(1,2).
p(2,3).
p(4,3).
p(`a',4).

q(x,y) :- p(x,y) and not x < y.

?-q(x,y).
\end{verbatim}
Produces this result set:
\begin{verbatim}p(4,3) p(`a',4)\end{verbatim}
As can be seen from this example, `not $X < Y$' is not the same as `$X \ge Y$'


\subsection{Arithmetic built-ins}

IRIS will automatically convert the result of an arithmetic evaluation to the most precise
type for both terms, e.g. a $double$ value + a $float$ value will result in a $double$, and a $float$ value + an $integer$ value will also result in a $double$.



\subsection{Custom built-in predicates}
\label{custom_builtins}
To create and use a custom built-in predicate there are a few steps to follow:

\begin{itemize}
\item Extend one of the built-in base classes (AbstractBuiltin, ArithmeticBuiltin, BooleanBuiltin)
\item Register the built-in with the BuiltinRegister class
\end{itemize}

\subsubsection{Extend one of the base classes}
There are 3 things that must be implemented:

\begin{enumerate}
\item a constructor taking an ITerm array as input that will contain the constants and variables occurring during evaluation
\item depending on which base class was extended, implement one of:
\begin{itemize}
\item AbstractBuiltin.evaluateTerms(ITerm[] terms, int[] variableIndexes)
\item BooleanBuiltin.computeResult(ITerm[] terms)
\item ArithmeticBuiltin.computeMissingTerm(int missingTermIndex, ITerm[] terms)
\end{itemize}
\item provide a static getBuiltinPredicate() method which returns the predicate object describing your built-in (with attributes
`name' and `arity')
\end{enumerate}

Note: The BuiltinHelper class has some utility methods that might be useful.
The javadoc for this class has more details.
For an example, see FahrenheitToCelsiusBuiltin.java in test/org.deri.iris.builtins.


\subsubsection{Register the built-in}
There are two ways to register a built-in:
\begin{itemize}
\item automatically by adding the class name to the `builtins.load' load file
\item  explicitly by calling the BuiltinRegister.registerBuiltin() method
\end{itemize}

To register the built-in automatically, the fully qualified name of the custom built-in java class
must be added to a file called `builtins.load'.
This file must be accessible through the classpath and IRIS will use the first
occurrence of a file with this name, ignoring any others.
Only one class name can occur on each line in the file.


To register the built-in explicitly, call BuiltinRegister.registerBuiltin() passing the built-in class's class object as the parameter, e.g.
\begin{verbatim}
<IProgram object>.getBuiltinRegister()
    .registerBuiltin(FahrenheitToCelsiusBuiltin.class);
\end{verbatim}
The built-in should now be recognised when you evaluate a logic program. Note that the built-in must
be explicitly registered for every program instance.


\section{API guide}

\subsection{Creating objects with the Java API}
\label{api}

Most of the objects created in IRIS are created with factories. There's a
factory for every purpose. The most important ones are described below:
\begin{description}
\item[org.deri.iris.api.factory.IProgramFactory] creates programs with or
without initial values.
\item[org.deri.iris.api.factory.IBasicFactory] creates tuples, atoms,
literals, rules and queries.
\item[org.deri.iris.api.factory.ITermFactory] creates variables, strings and
constructed terms.
\item[org.deri.iris.api.factory.IConcreteFactory] creates all other sorts of
terms (see section \ref{data} on page \pageref{data}).
\item[org.deri.iris.api.factory.IBuiltinsFactory] creates built-in atoms provided
by IRIS (see section \ref{builtins} on page \pageref{builtins}).
\end{description}

The \verb|org.deri.iris.factory.Factory| class holds \verb|static final|
instances of all the factories, so they can be easily \\
(e.g~\verb|import static org.deri.iris.factory.Factory.CONCRETE;|).

For a more complete list of methods, input parameters and return values it is
recommended to read the
javadoc\footnote{http://www.iris-reasoner.org/snapshot/javadoc/}.

\subsection{Creating objects using the parser}
\label{parser}

Instead of creating the java objects by hand, you could also use the \\
\verb|org.deri.iris.compiler.Paser| to parse a datalog program. The grammar
used by the parser is described in the grammar guide.

\subsection{Evaluating a program}

% will be written after the api change

\pagebreak
\appendix

\section{Datalog Grammar Support}
\label{grammar}
Datalog is a database query language that is syntactically a subset of Prolog.
Its origins date back to around 1978 when Herve Gallaire and Jack Minker
organized a workshop on logic and databases.
\footnote{http://en.wikipedia.org/wiki/Datalog}

\subsection{Datalog}

IRIS evaluates logic programs that contain rules and facts (the knowledge base) and
queries to be evaluated against this knowledge base.

All rules, facts and queries must be terminated by a `\verb|.|'.
\begin{description}
\item[rules]
consist of a head and a body. Both, the head and the body, are lists of
literals where the literals are separated by `\verb|,|' and the head and the body
are separated by `\verb|:-|'. The `\verb|,|' means `and', e.g.\\
~`\verb|ancestor(?X, ?Y) :- ancestor(?X, ?Z), ancestor(?Z, ?Y).|'
\item[facts]
are instances of predicates with constant terms, e.g.~`\verb|ancestor('john', 'odin').|'
\item[queries]
are literals prefixed with a `\verb|?-|'.
\item[literals]
 are positive or negative atoms `\verb|<atom>|' or `\verb|not <atom>|'.
\item[atoms]
 have the format `\verb|<predicate-symbol>(<terms>)|'. The terms must be
separated by `\verb|,|', e.g.~`\verb|ancestor('john', 'garfield')|'.
\item[terms]
are either constants or variables.
\item[variables]
are simple strings prefixed with a `\verb|?|', e.g.~'\verb|?VAR|'.
\end{description}

\subsection{Data types}
\label{grammar_datatypes}
The data types that IRIS supports are described in table \ref{datatypes} on page
\pageref{datatypes}.

\begin{table}[hbp]
\begin{tabular}{|l|l|}
\hline
\hline
datatype & syntax \\
\hline
\hline
string & \verb|'<string>'| \\
& \verb|_string('<string>')| \\
\hline
decimal & \verb"'-'?<integer>.<fraction>" \\
& \verb"_decimal('-'?<integer>.<fraction>)" \\
\hline
integer & \verb|'-'?<integer>| \\
& \verb|_integer('-'?<integer>)| \\
\hline
float &  \verb|_float(<integer>.<fraction>)| \\
\hline
double &  \verb|_double(<integer>.<fraction>)| \\
\hline
iri &  \verb|_'<iri>'| \\
&  \verb|_iri('<iri>')| \\
\hline
sqname &  \verb|<string>#<string>| \\
&  \verb|_sqname('<string>#<string>')| \\
\hline
boolean & \verb|_boolean(<string>)| \\
\hline
duration & \verb|_duration(<year>, <month>, <day>, <hour>, <minute>, | \\
& \hspace{1cm} \verb|<second>)| \\
& \verb|_duration(<year>, <month>, <day>, <hour>, <minute>, | \\
& \hspace{1cm} \verb|<second>, <millisec>)| \\
\hline
datetime & \verb|_datetime(<year>, <month>, <day>, <hour>, <minute>, | \\
& \hspace{1cm} \verb|<second>)| \\
& \verb|_datetime(<year>, <month>, <day>, <hour>, <minute>, | \\
& \hspace{1cm} \verb|<second>, <tzHour>, <tzMinute>)| \\
& \verb|_datetime(<year>, <month>, <day>, <hour>, <minute>, | \\
& \hspace{1cm} \verb|<second>, <millisec>, <tzHour>, <tzMinute>)| \\
\hline
date & \verb|_date(<year>, <month>, <day>)| \\
& \verb|_date(<year>, <month>, <day>, <tzHour>, <tzMinute>)| \\
\hline
time & \verb|_time(<hour>, <minute>, <second>)| \\
& \verb|_time(<hour>, <minute>, <second>, | \\
& \hspace{1cm} \verb|<tzHour>, <tzMinute>)| \\
& \verb|_time(<hour>, <minute>, <second>, <millisec>, | \\
& \hspace{1cm} \verb|<tzHour>, <tzMinute>)| \\
\hline
gyear & \verb|_gyear(<year>)| \\
\hline
gyearmonth & \verb|_gyearmonth(<year>, <month>)| \\
\hline
gmonth & \verb|_gmonth(<month>)| \\
\hline
gmonthday & \verb|_gmonthday(<month>, <day>)| \\
\hline
gday & \verb|_gday(<day>)| \\
\hline
hexbinary & \verb|_hexbinary(<hexbin>)| \\
\hline
base64binary & \verb|_base64binary(<base64binary>)| \\
\hline
\end{tabular}
\caption{All supported datatypes}
\label{datatypes}
\end{table}

\subsection{Built-in predicates}
\label{grammar_builtins}

A built-in can be written instead of a literal and IRIS will do its best to
evaluate it. All supported built-in predicates are described in table \ref{builtins} at
page \pageref{builtins}.

\begin{table}[hbp]
\begin{tabular}{|l|l|l|}
\hline
\hline
name & syntax & supported operations \\
\hline
\hline
add & \verb|?X + ?Y = ?Z| & numeric + numeric = numeric \\
& \verb|ADD(?X, ?Y, ?Z)| & date + duration = date \\
& & duration + date = date \\
& & time + duration = time \\
& & duration + time = time \\
& & datetime + duration = datetime \\
& & duration + datetime = datetime \\
& & duration + duration = duration \\
\hline
subtract & \verb|?X - ?Y = ?Z| & numeric - numeric = numeric \\
& \verb|SUBTRACT(?X, ?Y, ?Z)| & date - duration = date \\
& & date - date = duration \\
& & time - duration = time \\
& & time - time = duration \\
& & datetime - duration = datetime \\
& & datetime - datetime = duration \\
& & duration - duration = duration \\
\hline
multiply & \verb|?X * ?Y = ?Z| & numeric x numeric = numeric \\
& \verb|MULTIPLY(?X, ?Y, ?Z)| & \\
\hline
divide & \verb|?X / ?Y = ?Z| & numeric / numeric = numeric \\
& \verb|DIVIDE(?X, ?Y, ?Z)| & \\
\hline
equal & \verb|?X = ?Y| & any type = same type \\
& \verb|EQUAL(?X, ?Y)| & numeric = numeric\\
&  & any type = different type (always false)\\
\hline
not equal & \verb|?X != ?Y| & any type $ \ne $ same type \\
& \verb|NOT_EQUAL(?X, ?Y)| & numeric $ \ne $ numeric\\
&  & any type $ \ne $ different type (always true)\\
\hline
less & \verb|?X < ?Y| & any type $ < $ same type \\
& \verb|LESS(?X, ?Y)| & numeric type $ < $ numeric type \\
\hline
less-equal & \verb|?X <= ?Y| & any type $ \le $ same type \\
& \verb|LESS_EQUAL(?X, ?Y)| & numeric type $ \le $ numeric type \\
\hline
greater & \verb|?X > ?Y| & any type $>$ same type \\
& \verb|GREATER(?X, ?Y)| & numeric type $>$ numeric type \\
\hline
greater-equal & \verb|?X >= ?Y| & any type $ \ge $ same type \\
& \verb|GREATER_EQUAL(?X, ?Y)| & numeric type $ \ge $ numeric type \\
\hline
same type & \verb|SAME_TYPE(?X, ?Y)| & any type same\_type\_as any type \\
\hline
\end{tabular}
\caption{All supported binary and ternary built-in predicates}
\label{builtins}
\end{table}


\begin{table}[hbp]
\begin{tabular}{|l|l|l|}
\hline
\hline
name & syntax & supported operations \\
\hline
\hline
is base64binary & \verb|IS_BASE64BINARY(?X)| & true iff ?X is of type base64binary \\
\hline
is boolean & \verb|IS_BOOLEAN(?X)| & true iff ?X is of type boolean \\
\hline
is date & \verb|IS_DATE(?X)| & true iff ?X is of type date \\
\hline
is datetime & \verb|IS_DATETIME(?X)| & true iff ?X is of type datetime \\
\hline
is decimal & \verb|IS_DECIMAL(?X)| & true iff ?X is of type decimal \\
\hline
is double & \verb|IS_DOUBLE(?X)| & true iff ?X is of type decimal \\
\hline
is duration & \verb|IS_DURATION(?X)| & true iff ?X is of type duration \\
\hline
is float & \verb|IS_FLOAT(?X)| & true iff ?X is of type float \\
\hline
is gday & \verb|IS_GDAY(?X)| & true iff ?X is of type gday \\
\hline
is gmonth & \verb|IS_GMONTH(?X)| & true iff ?X is of type gmonth \\
\hline
is gmonthday & \verb|IS_GMONTHDAY(?X)| & true iff ?X is of type gmonthday \\
\hline
is gyear & \verb|IS_GYEAR(?X)| & true iff ?X is of type gyear \\
\hline
is gyearmonth & \verb|IS_GYEARMONTH(?X)| & true iff ?X is of type gyearmonth \\
\hline
is hexbinary & \verb|IS_HEXBINARY(?X)| & true iff ?X is of type hexbinary \\
\hline
is integer & \verb|IS_INTEGER(?X)| & true iff ?X is of type integer \\
\hline
is iri & \verb|IS_IRI(?X)| & true iff ?X is of type iri \\
\hline
is numeric & \verb|IS_NUMERIC(?X)| & true iff ?X is of any numeric type \\
 &  & (integer, float, double, decimal) \\
\hline
is sqname & \verb|IS_SQNAME(?X)| & true iff ?X is of type sqname \\
\hline
is string & \verb|IS_STRING(?X)| & true iff ?X is of type string \\
\hline
is time & \verb|IS_TIME(?X)| & true iff ?X is of type time \\
\hline
\end{tabular}
\caption{All supported unary built-in predicates}
\label{builtins}
\end{table}

Custom built-ins can also be registered, e.g. if a built-in with the
predicate symbol `\verb|ATOI|' is registered then the syntax will be
`\verb|ATOI(?MY_STRING, ?MY_INT)|'.

\end{document}
